\documentclass[../../rasd.tex]{subfiles}
\begin{document}

\section{Definitions, Acronyms, Abbreviations}
		\subsection{Definitions}
		\begin{description}
			\item[User]: registered individual of Data4Help who agreed on the acquisition and processing of their data (see \todo{add reference to user data below}).
			\item[User data]: \textit{User}'s health data and location acquired by Data4Help
			\item[Third party]: a company that can access \textit{User data} stored in TrackMe's database, after user consent.
			\item[Service]: application available for some Data4Help \textit{Users}, generally offered by a Third party.
			\item[Group data]: group of \textit{Users data} acquired by Data4Help.
			\item[Smart wearable]: smart devices that can be worn on the body as accessories. These devices are required to have specific sensors for data acquisition, to be compatible with the system to be (see \todo{add reference to requirements for smart wearables}). The adjective 'smart' refers to the possibility of connecting them to an external device, such as a smartphone, and to the ability of operating autonomously even if not connected.
			\item[Anomalous data]: health data that is outside certain intervals, which identify a \textit{User} normal health condition. \todo{define better}
			\item[User in need]: registered user of AutomatedSOS in need of assistance since their health data is \textit{amomalous}.
			\item[Run]: running competition registered on Track4Run. 
			\item[Organizer]: company or private person organizing a \textit{Run}.
			\item[Spectator]: person participating as spectator of a \textit{Run}.
			\item[Participant]: \textit{User} subscribed to Track4Run participating in a \textit{Run}.

		\end{description}
		\subsection{Acronyms}
		\begin{description}
		\item[GPS]: Global Positioning Service
		\end{description}
		\subsection{Abbreviations}
		\begin{description}
			\item[G\subs{n}]: n\sups{th} goal
			\item[D\subs{n}]: n\sups{th} domain assumption
			\item[R\subs{n}]: n\sups{th} requirement
			\item[S\subs{n}]: n\sups{th} scenario
			
		\end{description}
		
\end{document}