\providecommand{\rasd}{..}
\documentclass[../rasd.tex]{subfiles}

\begin{document}

\chapter{Introduction}
\thispagestyle{fancy}
		
		\subfile{\rasd/1.Introduction/1.Purpose/Purpose.tex}
		\subfile{\rasd/1.Introduction/2.Scope/Scope.tex}
		\subfile{\rasd/1.Introduction/3.Goals/Goals.tex}
		\subfile{\rasd/1.Introduction/4.DefinitionsAcronymsAbbreviations/DefinitionsAcronymsAbbreviations.tex}
	
		\section{Revision History}
		\begin{enumerate}
			\item Version 1.0 - 11 November 2018 - First Release
			\item Version 1.1 - 15 December 2018 - Second Release
				\paragraph{Major Changes}
				\begin{itemize}
					\item Added brief document purpose (sections \hyperref[sect:1.1]{1.1}).
					\item Definitions revision (section \hyperref[sect:1.4.1]{1.4.1}).
					\item Removed class Data Category from Data4Help Class Diagram (section \hyperref[sect:2.1.1]{2.1.1}).
					\item Added class Spectator to Track4Run Class Diagram (section \hyperref[sect:2.1.3]{2.1.3}).
					\item S\subs{3} revision (section \hyperref[sect:3.2.1]{3.2.1}).
					\item Add New Service Sequence Diagram revision (section \hyperref[sect:3.2.4]{3.2.4}).
					\item Requirements revision (section \hyperref[sect:3.2.5]{3.2.5}).
					\item Reliability further explained (section \hyperref[sect:3.5.1]{3.5.1}).
					\item Minor fixes
				\end{itemize}

		\end{enumerate}
		\section{Reference Documents}
			\begin{itemize}
				\item Rumbaugh, Jacobson, Booch. 1999. \ic{The Unified Modeling Language Reference Manual}. Addison-Wesley.
				\item MIT Software Design Group. \ic{Appendix B: Alloy Language Reference}. http://alloytools.org/documentation.html
				\item MIT Software Design Group. \ic{Tutorial Materials, Slides}. http://alloytools.org/tutorials/day-course/
			\end{itemize}
		\section{Document Structure}
		This document is divided into six chapters.
		\paragraph{Chapter 1}
		It introduces the project in terms of purpose, scope and goals. Moreover, it contains the definitions, acronyms and abbreviations needed to properly understand the sections following. All the documents used during the development of this project are listed.
		\paragraph{Chapter 2}
		This chapter goes deep into the system description and definition. In particular it has:
		\begin{itemize}
			\item A detailed description of the product to be delivered and its features.
			\item A list of all domain assumptions that enable to reach the goals described in section \hyperref[sect:1.3]{1.3}.
			\item A description of users for which the product was thought.
		\end{itemize}
		\paragraph{Chapter 3}
		It analyzes and lists all the requirements needed to reach the goals described in section \hyperref[sect:1.3]{1.3}. Requirements are a huge and fundamental part of the project life cycle. Therefore, the chapter is divided in several sections, one for each aspect of requirements definition:
		\begin{itemize}
			\item Functional Requirements are listed as a description of interfaces, scenarios and uses cases (use case and sequence diagrams are included).
			\item Non Functional Requirements: performance, reliability, availability, security, portability.
			\item Any design constraint.
		\end{itemize}
		\paragraph{Chapter 4}
		This chapter is dedicated to the formalization of the system and its scope through a formal representation of entities and constraints using Alloy. The representation is split between Data4Help, AutomatedSOS and Track4Run to keep the generated world as simple as possible and to keep the three systems separated even if they have some interaction. In fact, AutomatedSOS and Track4Run do not have interactions between each other, but they do with Data4Help. Therefore Data4Help appears in the world dedicated to the two \ic{Services}.
		\paragraph{Chapter 5}
		Effort spent by all team members is shown as the list of all activities done during the realization of this document.
		\paragraph{Chapter 6}
		References of documents that this project was developed upon.
		
\end{document}