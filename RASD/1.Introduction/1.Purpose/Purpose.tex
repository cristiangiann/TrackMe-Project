\documentclass[../../rasd.tex]{subfiles}
\begin{document}

\section{Purpose}
			
			TrackMe wants to offer a service named "Data4Help" on top of which will be built two services named "AutomatedSOS" and "Track4Run".
			\paragraph{Data4Help:} the basic idea behind Data4Help is to acquire the location and health data of \textit{Users} through \textit{smart wearables} connected to a smartphone. Moreover, data can be directly sent to \textit{Third Party} customers who pay for the service. In order to analyze \textit{Users data}, these need to obtain user authorization. Furthermore, they can request anonymized data of a group of users.
			
			\paragraph{AutomatedSOS:} a service offered only to subscribed customers that constantly monitors their health status. Its purpose is to identify when a \textit{User} is in need of immediate assistance and send an ambulance to their location.
			
			\paragraph{Track4Run:} a service used to track runners participating in running competitions. \textit{Organizers} will be able to define a path for the run, \textit{Participants} will share their position and health data  and \textit{Spectators} may watch the competition on their smart devices.
			
			\subsection{Goals}
				\subsubsection{Data4Help}
					\begin{itemize}
						%registration
						\item[G\subs{1}]Allow a person to register as User after his agreement of acquirement of data by TrackMe.
						\item[G\subs{2}]Allow a company to register as Third Party of Data4Help.
						%identification
						\item[G\subs{3}]Identify uniquely a User by their username and password.

						\item[G\subs{4}]Manage User data (\todo{add reference}) request from a Third Party.
							\begin{itemize}
								\item [G\subs{4.1}]Allow a Third Party to select a person whom want to access data through his fiscal code or his social security number.
								\item [G\subs{4.2}]Allow the specific User to accept or reject the data access request sent by the Third Party.
								\item [G\subs{4.3}]If the User accepts the request, their data is sent to the Third Party who made the request.
								\item [G\subs{4.4}]If the User does not accept the request, the Third Party who made the request is not allowed to see their data.
							\end{itemize}	
						\item [G\subs{5}]Manage group data (\todo{add reference}) request from a Third Party.
							\begin{itemize}
								\item [G\subs{5.1}]Allow a Third Party to specify one or more features to define a group of Users for which the request is made.
								\item [G\subs{5.2}]Accept a Third Party group data request if the group it refers to is made up of 1000 or more Users.
								\item [G\subs{5.3}]Reject a Third Party group data request if the group it refers to is made up of less than 1000 Users.
								\item [G\subs{5.4}]Guarantee proper anonymization of group data.

							\end{itemize}								
						\item[G\subs{6}]Allow a Third Party to access Users data as soon as they are produced.
						\item[G\subs{7}]Allow a Third Party to access only Users data whose permission was granted.
						
						%GDPR
						\item[G\subs{8}]Allow a User to revoke the previously given data access consent to any Third Party.
						\item[G\subs{9}]Retrieve all the data of a specific user when requested by them.
						\item[G\subs{10}]Delete all the data of a specific user when requested by them.

						\item[G\subs{11}]Monitor through smart wearables specific health parameters of Users registered to AutomatedSOS (see \todo{add reference to all the specific health parameters required by AutomatedSOS}).
					\end{itemize}



				\subsubsection{AutomatedSOS}
					\begin{itemize}
						%registration
						\item[G\subs{12}]Allow a Data4Help \textit{User} to subscribe to AutomatedSOS.

						\item[G\subs{13}]Analyze \textit{Users data} retrieved from Data4Help to possibly identify a \textit{User} as a \textit{User in need}. \todo{check this goal for better explanation}

						\item[G\subs{14}]Send an ambulance to the last position of a \textit{User in need}.
							\begin{itemize}
								\item[G\subs{14.1}]Retrieve the last position of a \textit{User in need}.
								\item[G\subs{14.2}]Make an emergency call to local emergency services within 5 seconds from the identification of a \textit{User in need}.
								\item[G\subs{14.3}]Send the last position of the \textit{User in need} to local emergency services.
								\item[G\subs{14.4}]Send \textit{User in need anomalous data} to local emergency services.
							\end{itemize}
					\end{itemize}




				\subsubsection{Track4Run}
					\begin{itemize}
						%registration
						\item[G\subs{15}]Allow an individual to register as \textit{Spectator} (see \todo{add reference to Spectator}).
						\item[G\subs{16}]Allow an individual to register as \textit{Organizer} (see \todo{add reference to Organizer}).
						\item[G\subs{17}]Allow a Data4Help \textit{User} to enroll in a run as a \textit{Participant} (see \todo{add reference to Participant}).
						%identification
						\item[G\subs{18}]Identify uniquely an Organizer by their username and password.
						
						%run creation
						\item[G\subs{19}]Allow \textit{Organizers} to create a run.
						\begin{itemize}
							\item[G\subs{19.1}]Allow \textit{Organizers} to give the run a name.
							\item[G\subs{19.2}]Allow \textit{Organizers} to set a date and time for the run.
							\item[G\subs{19.3}]Allow \textit{Organizers} to define a path for the run.
							\item[G\subs{19.4}]Let every run be identified by a unique \textit{run identifier}.
						\end{itemize}
						%premium functions for organizers
						\todo{add premium functions}

						%spectators
						\item[G\subs{20}]\textit{Spectators} can watch a run by knowing its \textit{run identifier} .
							\begin{itemize}
							\item[G\subs{20.1}]\textit{Spectators} can see the position of each \textit{Participant} on the map.
							\item[G\subs{20.2}]\textit{Spectators} can see health data of a specific \textit{Participant} on the map.
							\todo{check the verb 'see'}
							\end{itemize}
					\end{itemize}
\end{document}













