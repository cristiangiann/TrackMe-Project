\documentclass[../../rasd.tex]{subfiles}
\begin{document}

\section{Purpose}
			TrackMe offers a basic B2B service named "Data4Help" on top of which are built two B2C services named "AutomatedSOS" and "Track4Run".
			\paragraph{Data4Help:}It's a service that TrackMe company is willing to develop. The basic idea is allowing third parties to monitor the location and health status of individuals through many sensors. The service is based on the retrieval of data sent by registered users. Every user has one or more sensor that sends the information to TrackMe. Then, data can be directly sent to a third party client that pays for the service and had obtained the authorization of the user, or can contribute to a anonymous dataset (composed by at least a thousands people due to company policy). Users sending data are rewarded.\\
			
			\paragraph{AutomatedSOS:}It's a service that guarantees (within a certain amount of time - 5 seconds according company policy) the call of an ambulance if the health data that is received is under a given threshold.
			
			\paragraph{Track4Run:}It's a service that it can be used during a running competition: organizers can define a path and runners can enroll to it enabling spectators to track them in a map.
			\subsection{Goals}
			\begin{itemize}
				\item[G\subs{1}]Allow a person to register as Individual after his agreement of acquirement of data by TrackMe.
				
				\item[G\subs{2}]Allow a person or a company to register as third party of Data4Help.

				\item[G\subs{3}]Manage individual request of a third party.
					\begin{itemize}
					\item [G\subs{3.1}]	Allow a third party to select a person whom want to access data through his fiscal code or his social security number.
					\item [G\subs{3.2}]Allow the individual to accept or refuse the request.
					\item [G\subs{3.3}]If the Individual accept the request, his data are sent to the third party which made the request.
					\item [G\subs{3.4}]If the Individual does not accept the request, the third party which made the request is not able to see his data.
					\end{itemize}								
				\item [G\subs{4}]Manage groups of individuals request of a third party.\\
					\begin{itemize}
					\item [G\subs{4.1}]Allow a third party to select a group of people linked by one or more data.
					\item [G\subs{4.2}]If the request refers to 1000 individuals or more, the request is accepted and the data are anonymized before being sent to the third party which made the request.
					\item [G\subs{4.3}]If the request refers to less than 1000 individuals, the request is refused and the third party is not able to access to the data.
					\end{itemize}								
				\item[G\subs{5}]Allow to a third party to access to data of individuals of whom it have permission as soon as they are produced.
				\item[G\subs{6}]Allow to an Individual to revoke the availability of his data to a specific Third Party that has access to them.
				\item[G\subs{7}]Allow to elderly Individuals to subscribe to AutomatedSOS.
					
				\item[G\subs{8}]Monitor with smart devices the health parameters of Individuals registered to AutomatedSOS.
					
				\item[G\subs{9}]Send an ambulance to the position of an Individual registered to AutomatedSOS with health parameters beyond certain intervals.
					\begin{itemize}
						\item[G\subs{9.1}]If the health parameters registered of an Individual are beyond certain intervals, the system checks the position of the device.
						\item[G\subs{9.2}]The system makes an emergency call to the local emergency number explaining the position of the Individual and the values of his non-regular health parameters.
						\item[G\subs{9.3}]The emergency call is initiated within five seconds of the detection of the non-regular parameters.
					\end{itemize}


				\item WORK IN PROGRESS
			\end{itemize}
			
			
\end{document}