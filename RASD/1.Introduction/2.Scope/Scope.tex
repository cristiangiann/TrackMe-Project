\documentclass[../../rasd.tex]{subfiles}
\begin{document}

\section{Scope\label{sect:1.2}}
TrackMe offers its services in a world where technology and health are taking huge strides forward every day and innovation is commonplace.\\
Nowadays, people use smart devices such as smartphones and \ic{Smart wearables} more than any other object that they own. This means that any activity they perform can be integrated with these devices.\\\\
TrackMe, with the introduction of Data4Help, offers the possibility to monitor \ic{Users}’ location and health data and allows \ic{Third parties} to acquire these data.\\\\
When it comes to personal data acquisition, privacy is a fundamental issue that TrackMe needs to consider. Privacy is, in fact, regulated by several laws: there are many restrictions on how \ic{User data} is acquired and stored. Therefore, TrackMe is concerned with \ic{Users}’ consent to transferring data to TrackMe itself and to third parties for individual specific analysis. Moreover, TrackMe guarantees that data of groups of individuals (\ic{Group data}) are properly anonymized by checking specific constraints.\\\\
Over the course of their daily routine, \ic{Users} perform several actions during which their data can be analyzed to provide them with insights. For instance, they might want to monitor their heart rate while sleeping or to keep track of the distance they have walked during their day and the places they have been to.\\\\
People with a potential need for immediate assistance have always been a huge concern for their relatives and technology makers. These may include old people with limited movement and a high chance to need urgent assistance, anyone who has a specific disease, but also a healthy individual who may suffer from a sudden heart failure. Until now, the only practical way to receive help has been calling for help, either by using a cell phone or by pushing an SOS button on a dedicated device. TrackMe proposes to automatize the step of calling for help through AutomatedSOS. In fact, when determined health values will no more be considered as normal, the system will automatically send a request for help.\\\\
Furthermore, nowadays, when it comes to sports and working out, having the possibility of collecting and sharing athletes' data is a disruptive innovation. In fact, giving anybody the possibility of having on their smartphone an accurate analysis of their health while performing a work out session is a breakthrough.\\
A sport that is practiced and loved by many is running. Organizing a run requires several steps to be taken such as defining a path, getting athletes to participate and spectators to watch it. TrackMe proposes to simplify the organization of a run, by introducing Track4Run. This service will allow the definition of a path, easy enrollment for participants and a real-time tracking of each runner’s position on a map.


\subsection{Analysis of Shared Phenomena}


\begin{enumerate}
	\item \ic{Users} register to Data4Help.
	\item \ic{Users} move.
	\item \ic{Users} can have health problems.
	\item \ic{Smart wearable} sensors acquire data.
	\item \ic{Smart wearables} communicate with Data4Help through smartphones.
	\item \ic{Third parties} register to Data4Help.
	\item \ic{Third parties} collect data from Data4Help.
	\item \ic{Users} grant direct usage and sharing of their personal data.
	\item \ic{Users} add a new \ic{Service}.
	\item \ic{Organizers} define a path for a \ic{Run}.
	\item \ic{Participants} enroll in a \ic{Run}.
	\item \ic{Spectators} of a \ic{Run} see on a map the position of the runners.
\end{enumerate}

\end{document}