\documentclass[../../../rasd.tex]{subfiles}
\begin{document}

\subsection{Scenarios}
        \paragraph{Data4Help}
        \begin{itemize}

                    %user registration and Login                
                    \item[S\subs{1}] Dante is an individual who would like to keep track of his GPS position and health data. For this purpose he decides to use Data4Help. He downloads the Data4Help application on his smartphone and proceeds to sign up. He inserts all required information, which include his name, his social security number and date of birth. He is asked to insert an email that will later be his \ic{Username} and a password. Dante inserts his name as his password and the system tells him that the inserted password is shorter than 8 characters, so he tries again with a new one. Eventually he inserts a valid password, accepts the \ic{Terms and conditions} and taps on "Create an Account". He is successfully signed up, after receiving a confirmation email by TrackMe. He tries to Loginto the application by inserting the newly created \ic{Username} and password. The system accepts the credentials and Dante is in.
                    
                    %Third Party registration and Login                
                    \item[S\subs{2}] YourHealth is a company that analyzes individuals' health data to provide users with insights on their well-being. It decides to offer its service also on Data4Help so as to have a greater pool of users. The person in charge navigates to the \ic{Third Party} dedicated website and clicks on "Sign Up". They fill in all required information about their company, insert an email that will be used as \ic{Username} and a valid password. They then accept the \ic{Terms and conditions} by clicking the specific checkbox, and finally click on "Create an Account". YourHealth receives an email confirming the account creation: now YourHealth \ic{Services} are available to all Data4Help \ic{Users}.
                    
                    %User adds a service and consequently Third Party requests user data
                    \item[S\subs{3}] Dante, a Data4Help \ic{User}, needs to monitor his heart rate through the day. He navigates to the "Discover" page inside of his Data4Help application and scrolls through the available \ic{Services}. He finds MyHeart, a \ic{Service} developed by YourHealth, a \ic{Third Party} registered to Data4Help. The description of the service seems to suit his need, so he adds MyHeart to his \ic{Services}. In order to finalize the subscription, Dante will have to accept that his data will be sent to YourHealth for analysis. He does so. After a while, in the specific MyHeart \ic{Service} page, the "Analyze" button appears. Dante taps on it and promptly he sees a personalized graph showing his heart rate levels throughout the day, starting from the first day he registered to Data4Help.

                    %Third Party requests group data
                    \item[S\subs{4}] LocalStats is a company that performs intensive statistics on individuals' positions in some cities of Switzerland. It decides to acquire individuals' GPS positions data from Data4Help to enlarge its database. LocalStats registers as a Data4Help \ic{Third Party}. Once registration is complete, the first request it makes to Data4Help refers to all female \ic{Users} between 30 and 35 years old living in Lausanne. Unfortunately, the number of \ic{Users} with the requested characteristics is less than 1000, which does not guarantee proper data anonymization. Therefore, Data4Help rejects the \ic{Group Data} request. LocalStats tries again changing the interval of interest to 25-35 years old. This time the request refers to more than 1000 \ic{Users} and finally Data4Help can send the requested \ic{Group Data} to LocalStats.

                    %Third Party subscribes to new data
                    \item[S\subs{5}] Dante, from scenario S\subs{3}, would like to keep MyHeart active day by day. To do so, he taps on "Analyze Daily", which is a function offered by MyHeart. YourHealth, which developed and manages MyHeart, requests subscription to Dante's new data. Data4Help registers that anytime Dante's \ic{User data} is collected, it needs to send it to YourHealth for analysis. Starting from the following day, Dante does not need anymore to tap on "Analyze" every day: new analysis is provided to him as soon as it is available from MyHeart.

                    %user revokes consent to a Third Party service
                    \item[S\subs{6}] Dante, who subscribed to Data4Help and used its \ic{Third Party Services} for a while, decides that he does not want to use one of them, TrackKer, anymore. Therefore, he navigates to the "My Services" page and taps on TrackKer. The \ic{Service} page shows up and he taps on the "Revoke Consent" button at the bottom of the page. From now on, Data4Help will stop sending Dante's data to the \ic{Third Party} managing TrackKer.
                \end{itemize}

                \paragraph{AutomatedSOS}
                \begin{itemize}
                    %register anomalous data for a user, collect location and call ambulance                
                    \item[S\subs{7}] GianVito is a 57 years old man subscribed to AutomatedSOS. After getting very angry at work, he drives home, but as soon as he gets there he feels dizzy and falls on the ground. He is alone and cannot call for help. Fortunately, AutomatedSOS notices that his heart rate is below a certain threshold and identifies him as \ic{User in need}. AutomatedSOS calls local emergency services and sends them GianVito's position and health data. When the local emergency services dispatch an ambulance and GianVito is being taken care of, AutomatedSOS waits for GianVito's action, still collecting his data, without possibly identifying it as \ic{Anomalous}. Finally, GianVito opens up Data4Help application and taps on "Reactivate Monitoring". AutomatedSOS has fulfilled his need for immediate assistance and starts monitoring his health data again.
                \end{itemize}

                \paragraph{Track4Run}
                \begin{itemize}
                    %organize run                
                    \item[S\subs{8}] Charity4All is a Swedish charity association that organizes a running competition every year to raise money for their causes. The person in charge decides to use Track4Run to manage the run. They navigate to the \ic{Run} dedicated website and sign up as an \ic{Organizer}, inserting an email and a password for registration. Once sign up is complete, they click on "Create Run" and the \ic{Run} creation page shows up. They give the \ic{Run} a name - Run4Char - they define a path around Gothenburg and set the date and time the competition will take place on. They do not want to limit the number of participants, so they click on "Create Run" and obtain a \ic{Run} identifier back from Track4Run. They will distribute this identifier to all viewers who wish to enjoy the \ic{Run} on their devices.

                    %participant enrolls in run
                    \item[S\subs{9}] Hannah lives in Gothenburg and she loves running. In fact, she is subscribed to Track4Run. While browsing the available \ic{Runs} in her city, she finds Run4Char - from S\subs{8}. She enrolls in the run right away and Track4Me records her registration. Hannah is now a \ic{Participant} of the \ic{Run}.  

                    %spectator wants to watch run
                    \item[S\subs{10}] George enjoys sports a lot, however he is very old now and cannot participate in competitions anymore. He still likes watching sports event, especially when it comes to running. Since he is also into helping others, he is subscribed to Charity4All - from S\subs{8} - newsletter. He reads that they are organizing a \ic{Run} and writes down the \ic{Run} identifier. On the day of the \ic{Run}, he navigates to the \ic{Spectators} dedicated website and inserts the \ic{Run} identifier. As soon as the \ic{Run} starts, he enjoys it by watching the position of the \ic{Participants} on the map right on his device, comfortably in his house.
                \end{itemize}

\end{document}