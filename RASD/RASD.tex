\documentclass{report}

\title{
\small Politecnico di Milano\\
\small AA 2018-2019\\
\large Computer Science and Engineering\\
\Large Software Engineering 2 Project\\
\huge Requirement Analysis and Specification Document
}
\date{2018-11-11}
\author{
\large Gargano Jacopo Pio, Giannetti Cristian, Haag Federico}

\begin{document}
	\pagenumbering{gobble}
	\maketitle
	
	\newpage
	\pagenumbering{arabic}
	\tableofcontents
	
	\newpage
	\chapter{Introduction}
		\section{Purpose}
			{\bf Data4Help } is a service that TrackMe company is willing to develop. The basic idea is allowing third parties to monitor the location and health status of individuals through many sensors. The service is based on the retrieval of data sent by registered users. Every user has one or more sensor that sends the information to TrackMe. Then, data can be directly sent to a third party client that pays for the service and had obtained the authorization of the user, or can contirbute to a anonymous dataset (composed by at least a thousands people due to company policy).\\
			On top of Data4Help's infrastructure are built two products that are offered by TrackMe: AutomatedSOS and Track4Run.\\
			\\{\bf AutomatedSOS:} It's a service that guarantees (within a certain amount of time - 5 seconds according company policy) the call of an ambulance if the health data that is received is under a given threshold. \\\\{\bf Track4Run:} It's a service that it can be used during a running competition: organizers can define a path and runners can enroll to it enabling spectators to track them in a map.
			\subsection{Goals}
			TO DO LIST OF GOALS
		\section{Scope}
			TO DO SUMMARY OF THE WORLD (SCOPE)
			\subsection{Analysis of shared phenomena}
			TO DO LIST OF SHARED PHENOMENA
			
		\begin{enumerate}
		\item users move (or run in Track4Run)
		\item users can have health problems
		\item sensors collect data
		\item sensors communication
		\item sensors break
		\item third parties collect data from the system
		\item third parties registration to data4help
		\item user grant direct usage of personal data
		\item user registration (data4help and/or services built on top of it)
		\item organizers of run define path
		\item partecipants of run enroll to it
		\item run spectators see on a map the position of runners
		
		\end{enumerate}
		\section{Definitions, Acronyms, Abbreviations}
		TO DO DURING THE WRITING OF THIS DOCUMENT
		\begin{itemize}
		\item Third Parties: companies that want to buy people's sensors' data
		\item Wearable: ... TODO ...
		\end{itemize}
		\section{Revision history}
		\begin{enumerate}
			\item v. 1.0 - ????
		\end{enumerate}
		\section{Reference Documents}
		TO DO DURING THE WRITING OF THIS DOCUMENT
		\section{Document Structure}
		WORK IN PROGRESS
	\chapter{Overall description}
		\section{Product perspective}
			WORK IN PROGRESS

			\subsection{Goals}
			\begin{itemize}
				\item[G1] Allow a person to register as Individual after his agreement of acquirement of data by TrackMe.
				
				\item[G2] Allow a person or a company to register as Third Part of Data4Help.

				\item[G3] Manage individual request of a Third Part.\\
				$\left[G3.1\right]$ Allow a Third Part to select a person whom want to access data through his fiscal code or his social security number.\\
				$\left[G3.2\right]$ Allow the Individual to accept or refuse the request.\\
				$\left[G3.3\right]$ If the Individual accept the request, his data are sent to the Third Part which made the request.\\
				$\left[G3.4\right]$ If the Individual does not accept the request, the Third Part which made the request is not able to see his data.
				
				\item[G4] Manage groups of individuals request of a Third Part.\\
				$\left[G4.1\right]$ Allow a Third Part to select a group of people linked by one or more data.\\
				$\left[G4.2\right]$ If the request refers to 1000 Individuals or more, the request is accepted and the data are anonymized before being sent to the Third Part which made the request.\\
				$\left[G4.2\right]$ If the request refers to less than 1000 Individuals, the request is refused and the Third Part is not able to access to the data.
				
				\item[G5] Allow to a Third Part to access to data of Individuals of whom it have permission as soon as they are produced
				
			WORK IN PROGRESS
			\end{itemize}

		\section{Product perspective}
		WORK IN PROGRESS
		\section{Product functions}
		WORK IN PROGRESS
		\section{User characteristics}
		WORK IN PROGRESS
		\section{Assumptions, dependencies and constraints}
		WORK IN PROGRESS
	\chapter{Specific requirements}
		\section{External Interface Requirements}
			\subsection{User Interfaces}
			WORK IN PROGRESS
			\subsection{Hardware Interfaces}
			WORK IN PROGRESS
			\subsection{Software Interfaces}
			WORK IN PROGRESS
			\subsection{Communication Interfaces}
			WORK IN PROGRESS
		\section{Functional Requirements}
		WORK IN PROGRESS
		\section{Performance Requirements}
		WORK IN PROGRESS
		\section{Design Constraints}
			\subsection{Standards compliance}
			WORK IN PROGRESS
			\subsection{Hardware limitations}
			WORK IN PROGRESS
			\subsection{Any other constraint}
			WORK IN PROGRESS
		\section{Software System Attributes}
			\subsection{Reliability}
			WORK IN PROGRESS
			\subsection{Availability}
			WORK IN PROGRESS
			\subsection{Security}
			WORK IN PROGRESS
			\subsection{Maintainability}
			WORK IN PROGRESS
			\subsection{Portability}
			WORK IN PROGRESS
	\chapter{Formal analysis using Alloy}
	WORK IN PROGRESS
	\chapter{Effort spent}
	WORK IN PROGRESS
	\chapter{References}
	WORK IN PROGRESS
	
\end{document}