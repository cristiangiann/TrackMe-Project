\documentclass{report}

\title{
\small Politecnico di Milano\\
\small AA 2018-2019\\
\large Computer Science and Engineering\\
\Large Software Engineering 2 Project\\
\huge Requirement Analysis and Specification Document
}
\date{2018-11-11}
\author{
\large Gargano Jacopo Pio, Giannetti Cristian, Haag Federico}

\begin{document}
	\pagenumbering{gobble}
	\maketitle
	
	\newpage
	\pagenumbering{arabic}
	\tableofcontents
	
	\newpage
	\chapter{Introduction}
		\section{Purpose}
			TrackMe offers a basic B2B service named "Data4Help" on top of which are built two B2C services named "AutomatedSOS" and "Track4Run".
			\paragraph{Data4Help:} It's a service that TrackMe company is willing to develop. The basic idea is allowing third parties to monitor the location and health status of individuals through many sensors. The service is based on the retrieval of data sent by registered users. Every user has one or more sensor that sends the information to TrackMe. Then, data can be directly sent to a third party client that pays for the service and had obtained the authorization of the user, or can contribute to a anonymous dataset (composed by at least a thousands people due to company policy). Users sending data are rewarded.\\
			
			\paragraph{AutomatedSOS:}It's a service that guarantees (within a certain amount of time - 5 seconds according company policy) the call of an ambulance if the health data that is received is under a given threshold.
			
			\paragraph{Track4Run:} It's a service that it can be used during a running competition: organizers can define a path and runners can enroll to it enabling spectators to track them in a map.
			\subsection{Goals}
				\begin{itemize}
					\item[G1] Allow a person to register as Individual after his agreement of acquirement of data by TrackMe.
				
					\item[G2] Allow a person or a company to register as Third Party of Data4Help.

					\item[G3] Manage individual request of a Third Party.\\
					$\left[G3.1\right]$ Allow a Third Party to select a person whom want to access data through his fiscal code or his social security number.\\
					$\left[G3.2\right]$ Allow the Individual to accept or refuse the request.\\
					$\left[G3.3\right]$ If the Individual accept the request, the Third Party which made it is allowed to access to his data.\\
					$\left[G3.4\right]$ If the Individual does not accept the request, the Third Party which made the request is not able to see his data.
				
					\item[G4] Manage groups of individuals request of a Third Party.\\
					$\left[G4.1\right]$ Allow a Third Party to select a group of people linked by one or more data.\\
					$\left[G4.2\right]$ If the request refers to 1000 Individuals or more, the request is accepted and the data are anonymized before being sent to the Third Party which made the request.\\
					$\left[G4.2\right]$ If the request refers to less than 1000 Individuals, the request is refused and the Third Party is not able to access to the data.
				
					\item[G5] Allow to a Third Party to access to data of Individuals of whom it have permission as soon as they are produced.

					\item[G6] Allow to an Individual to revoke the availability of his data to a specific Third Party that has access to them.
					
					\item[G7] Allow to elderly Individuals to subscribe to AutomatedSOS.
					
					\item[G8] Monitor with smart devices the health parameters of Individuals registered to AutomatedSOS.
					
					\item[G9] Send an ambulance to the position of an Individual registered to AutomatedSOS with health parameters beyond certain intervals.\\
					$\left[G9.1\right]$ If the health parameters registered of an Individual are beyond certain intervals, the system checks the position of the device.\\
					$\left[G9.2\right]$ The system makes an emergency call to the local emergency number explaining the position of the Individual and the values of his non-regular health parameters.\\
					$\left[G9.3\right]$ The emergency call is initiated within five seconds of the detection of the non-regular parameters.\\
				
					\item WORK IN PROGRESS
				\end{itemize}
			
		\section{Scope}
			TrackMe offers its services in a world where technology and health are taking huge strides forward every day and innovation is commonplace.\\
Nowadays, people use smart devices such as smartphones and smart wearables more than any other object that they own. This means that any activity they perform already is or can be integrated with these devices.\\
TrackMe, with the introduction of Data4Help, offers the possibility to monitor users’ location and health data and allows third parties to register in the system to acquire these data.\\
When it comes to personal data acquisition, privacy is a fundamental issue that TrackMe needs to consider. Privacy is, in fact, regulated by several laws: there are many restrictions on how user’s data is acquired and stored. Therefore, TrackMe is concerned with users’ consent to transferring data to TrackMe itself and to third parties for individual specific analysis. Moreover, TrackMe guarantees that anonymized data of groups of individuals are properly anonymized by checking specific constraints.\\

Over the course of their daily routine, users perform several actions during which their data can be analyzed to provide them with insights. For instance, they might want to monitor their heart rate while sleeping or to keep track of the distance they have walked during their day and the places they have been to.\\
People with a potential need for immediate assistance have always been a huge concern for their relatives and for technology makers. These may include old people with limited movement and a high chance to need urgent assistance, anyone who has a specific disease, but also a healthy individual who can suffer from a sudden heart failure. Until now, the only practical way to receive help has been to call for help, either by using a cell phone or by pushing an SOS button on a dedicated device. TrackMe proposes to automatize the step of calling for help through AutomatedSOS. In fact, when determined health values will no more be considered as normal, the system will automatically send a request for help.\\
Moreover, a considerable percentage of people work out on a daily basis or simply enjoy being fit through some physical exercise sporadically. Personal trainers can be helpful in providing tips and work out routines that people can follow. Doctors should periodically monitor an individual’s health status, especially if they exercise. However, these not only are expensive but also, most importantly, aren’t always available.\\
A sport practiced and loved by many is running. Organizing a run requires several steps to be taken such as defining a path, getting athletes to participate and spectators to watch it. TrackMe proposes to simplify the organization of a run, by introducing Track4Run. This service will allow the definition of a path, easy enrollment for participants and a real-time tracking of each runner’s position on a map.
			\subsection{Analysis of shared phenomena}
			TO DO LIST OF SHARED PHENOMENA
			
		\begin{enumerate}
		\item users move (or run in Track4Run)
		\item users can have health problems
		\item sensors collect data
		\item sensors communication
		\item sensors break
		\item third parties collect data from the system
		\item third parties registration to data4help
		\item user grant direct usage of personal data
		\item user registration (data4help and/or services built on top of it)
		\item organizers of run define path
		\item partecipants of run enroll to it
		\item run spectators see on a map the position of runners
		
		\end{enumerate}
		\section{Definitions, Acronyms, Abbreviations}
		\begin{description}
			\item[Data-senders]: people using Track4Help sending data from device(s)
			\item[Third parties]: companies or private persons retrieving data from TrackMe
			\item[Organizers]: companies or private persons organizing running competitions
			\item[Spectators]: people participating as spectators to running competitions
			\item[Participants]: people running in running competitions
		\end{description}
		
		\section{Revision history}
		\begin{enumerate}
			\item v. 1.0 - ????
		\end{enumerate}
		\section{Reference Documents}
		TO DO DURING THE WRITING OF THIS DOCUMENT
		\section{Document Structure}
		WORK IN PROGRESS
	\chapter{Overall description}
		\section{Product perspective}
			WORK IN PROGRESS

		\section{Product perspective}
		WORK IN PROGRESS
		\section{Product functions}
		WORK IN PROGRESS
		\section{User characteristics}
			\subsection{Data4Help}
			\paragraph{Data-senders:}
			People having at least one device with a sensor connected to internet, willing to share the collected data with TrackMe to take advantage of the company rewards and eventually use the services built on top of Data4Help service.
			\paragraph{Third parties:}
			Companies or private persons willing, for any reasons, to collect bulk data. Usually the data is used for building services on top of it; in this case it is very important that data is transfered real time. Sometimes it is used just for statistics analysis. In both cases, Third parties need that the data is correct and precise. 
			\subsection{AutomatedSOS}
			\paragraph{Data-senders:}
			People (mainly elderly) having some disease or high risk of disorder willing to monitor own health parameters and conditions in order to prevent and possibly avoid crisis.
			\subsection{Track4Run}
			\paragraph{Participants:}
			People participating in running competition. Need to have small device and with no interaction during the run to avoid distractions.
			\paragraph{Organizers:}
			Companies or private persons organizing running competitions willing to better engage the spectators giving the possibility to track in real-time the position of all participants. Need to provide the service easily to ensure spectators are not prevented in using it.
			\paragraph{Spectators:}
			People participating as spectators to running competitions, willing to better engage the event by tracking the runners during all the run. Want it to be easy without needed particular devices or applications. Usually it is a "one-time usage".
		\section{Assumptions, dependencies and constraints}
			\subsection{Assumptions}
			 	\begin{itemize}
				 	\item Personal data inserted by the user during the registration correspond to his real data
					\item Data collected and sent by the GPS and the sensors at a certain instant correspond to the status of the user at that time
				
					\item User is logged in one device at a time

					\item The maps used faithfully represent the streets that can be traveled
					
				\end{itemize}
					
				
			
		WORK IN PROGRESS
	\chapter{Specific requirements}
		\section{External Interface Requirements}
			\subsection{User Interfaces}
			WORK IN PROGRESS
			\subsection{Hardware Interfaces}
			WORK IN PROGRESS
			\subsection{Software Interfaces}
			WORK IN PROGRESS
			\subsection{Communication Interfaces}
			WORK IN PROGRESS
		\section{Functional Requirements}
		WORK IN PROGRESS
		\section{Performance Requirements}
		WORK IN PROGRESS
		\section{Design Constraints}
			\subsection{Standards compliance}
			WORK IN PROGRESS
			\subsection{Hardware limitations}
			WORK IN PROGRESS
			\subsection{Any other constraint}
			WORK IN PROGRESS
		\section{Software System Attributes}
			\subsection{Reliability}
			WORK IN PROGRESS
			\subsection{Availability}
			WORK IN PROGRESS
			\subsection{Security}
			WORK IN PROGRESS
			\subsection{Maintainability}
			WORK IN PROGRESS
			\subsection{Portability}
			WORK IN PROGRESS
	\chapter{Formal analysis using Alloy}
	WORK IN PROGRESS
	\chapter{Effort spent}
	WORK IN PROGRESS
	\chapter{References}
	WORK IN PROGRESS
	
\end{document}