\documentclass[../../rasd.tex]{subfiles}
\begin{document}

\section{Product functions}
			\subsection{Data4Help}
			
				\subsubsection{User Registration}
				Data4Help will allow users to register. These will register by entering all the required information (see [R\subs{x}] \todo{add reference to Requirements where we specify User info for registration}). When registering to Data4Help, a user will first declare to have read the privacy statement and second they will have to accept the terms and conditions, which specifically include their consent to the acquisition and processing of their data, including sensitive ones, by TrackMe.\\
				The user registration process will be carried out on the user dedicated application (see \todo{add reference to where we are going to specify the user interface - probably section User Interface (mobile app), maybe add also reference to user registration scenario / use case}.\\
				Unregistered users will not be able to use Data4Help. (\todo{maybe this is a requirement and it is not necessary to put it here})
			
				\subsubsection{Third Party Registration}
				A third party will be automatically registered to Data4Help once the (\todo{add name of contract between third parties and TrackMe, both here and in the definitions}) contract has been agreed upon and signed by both TrackMe and the third party itself.\\
				Third party registration is required for using the service: an unregistered third party must not be able to access users data.(\todo{maybe this is a requirement and it is not necessary to put it here})
				When registering, third parties specify which services built on top of Data4Help they would like to use.
			
				\subsubsection{User Data Acquisition}
				Data4Help will acquire users data through smart wearables. \\
				Users must give consent to the acquisition of their data when registering to Data4Help.\\
				Data acquisition frequency can be changed according to users or third parties needs. For instance, if a user wants to save their smart wearable battery, frequency can be reduced. On the other hand, if a third party would like to track more accurately a user's position, a higher location acquisition frequency can be requested.	

				\subsubsection{Third Parties Data Access}
				Once a third party is registered to Data4Help, it can request access to users data acquired through Data4Help and stored by TrackMe. Third parties may request data that refers either to a specific individual or to a group of users.\\
				Consent to individual data access is left to the specific user, who can either allow or deny a third party request.\\
				Data on groups of individuals will be shared with third parties as long as TrackMe will be able to anonymize it properly (see R\subs{x} \todo{include reference to requirement about anonymized data (1000 users satisfy request}).

				\subsubsection{Data Management and Privacy}
				All data acquired through Data4Help will be stored on a database accessible only by TrackMe. Each piece of users data will have a list of parties to whom access was granted. At any time, a user will be able to revoke the previously given consent to any third party or to TrackMe. Moreover, a user may exercise their right to data portability, which means that TrackMe will have to provide them with all the collected data regarding them. Finally, users may ask the deletion of all their data stored by TrackMe. \todo{might want to list the requirements that relate to this}\\
				By guaranteeing these functions, Data4Help will respect existing general regulations on privacy (e.g. EU GDPR). 
			
			\subsection{AutomatedSOS}
				\subsubsection{User Registration}
				\subsubsection{Health Status Monitoring}
				\subsubsection{Ambulance Dispatching}

			\subsection{Track4Run}
				\subsubsection{User Registration}
				Track4Run will be a service used by three different kinds of users: organizers, runners and spectators. All of them need to register to the service for different purposes explained below.
				\begin{itemize}
					\item[Organizers]
					Runs can be organized by users defined as organizers. They will simply have to open the Track4Run dedicated application (see \todo{include reference to mobile app T4R}) and

					\todo {Understand whether or not to impolement two different apps? --> in my opinion we should implement only 1 app and have a section dedicated to track4run. By doing this we can exploit the registration process offered by Data4help}
					\item[Runners]
					\item[Spectators]
					\todo{these probably don't need to login, maybe we can have a "I'm a spectator button in the login section, where they can enter the run code (just like kahoot. Or if they want to they can be logged in data4help and do it (in the T4R section))"}
				\end{itemize}
				\subsubsection{Run Creation and Path Definition}
				Organizers have the ability of creating a run. They will be able to give the run a name, set a date the run is going to be held on and define a path for the run. Premium functions will be included in Track4Run. One of them will be sending an invitation to specific participants via email. Another premium function will allow organizers to make their run public and let anybody enroll to it or be a spectator of it. Finally premium users will be able to put constraints to specific parameters of the runners, including but not limited to age, sex and weight.
				\subsubsection{Runners Map Display}
				\todo{might want to define how the map is implemented, maybe it uses googlemaps, but especially what is its main purpose and what can spectators do on it. \\ for example they could be able to follow a specific runner, see health data of a runner}





\end{document}