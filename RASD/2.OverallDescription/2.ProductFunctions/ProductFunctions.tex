\documentclass[../../rasd.tex]{subfiles}

\begin{document}

\section{Product functions}
			\subsection{Data4Help}
			
				\subsubsection{User Registration}
				Data4Help will allow individuals to register. These will register by entering all the required information (see R\subs{2}). When registering to Data4Help, an individual will first declare to have read the privacy statement and secondly they will have to accept the terms and conditions, which specifically include their consent to the acquisition and processing of their data, including sensitive ones, by TrackMe.\\
				The \ic{User} registration process will be carried out on the Data4Help application (see Section 3.1.1).\\
			
				\subsubsection{Third party Registration}
				A \ic{Third party} may register to Data4Help through the \ic{Third party} dedicated website, including all required information (see R\subs{3}).
				Once terms and conditions have been accepted by the \ic{Third party}, it will be successfully registered to the service.\\
							
				\subsubsection{User Data Acquisition}
				Data4Help will acquire \ic{User data} through \ic{Smart wearables}. \\
				\ic{Users} must give consent to the acquisition of their data when registering to Data4Help.\\
				Data acquisition frequency can be changed according to \ic{Third party} needs. For instance, if a \ic{Third party} would like to track more accurately the position of a \ic{User}, a higher location acquisition frequency can be requested.

				\subsubsection{Third party Data Request}
				Once a \ic{Third party} is registered to Data4Help, it can request access to \ic{Users data} acquired through Data4Help and stored by TrackMe. \ic{Third parties} may request data that refers either to a specific individual - \ic{User data} - or to a group of \ic{Users} identified by common characteristics - \ic{Group data}.\\
				Consent to individual data access is left to the specific \ic{User}, who can either give or deny it to a \ic{Third party} request.\\
				\ic{Group data} will be shared with \ic{Third parties} as long as TrackMe will be able to anonymize it properly (see R\subs{23}).

				\subsubsection{Data Management and Privacy}
				All data acquired through Data4Help will be stored on a database accessible only by TrackMe. Each piece of \ic{Users data} will have a list of \ic{Services} offered by \ic{Third parties} to whom access was granted by the \ic{User}. At any time, a \ic{User} will be able to revoke the previously given consent to any \ic{Third party} or to TrackMe. Moreover, a \ic{User} may exercise their right to data portability, which means that TrackMe will have to provide them with all the collected data regarding them (see R\subs{32} and R\subs{34}). Finally, \ic{Users} may ask the deletion of all their data stored by TrackMe (see R\subs{33} and R\subs{35}).\\
				By guaranteeing these functions, Data4Help will respect existing general regulations on privacy (e.g. EU GDPR).
			
			\subsection{AutomatedSOS}
				
				\subsubsection{User Subscription}
				All Data4Help \ic{Users} may subscribe to AutomatedSOS through Data4Help application (see Section 3.1.1).
				
				\subsubsection{Health Status Monitoring}
				The service will constantly monitor \ic{User}'s health data to verify if it is \ic{Anomalous}. While data acquisition frequency can be tweaked only by Data4Help, AutomatedSOS may request a different value according to user needs.
				
				\subsubsection{Calling an Ambulance}
				In case the health status of a subscribed \ic{User} is considered not to be good, AutomatedSOS will make a call to local emergency services within 5 seconds and send an ambulance to the last registered location of the \ic{User}.

			\subsection{Track4Run}
				
				\subsubsection{User Registration}
				Track4Run will be a service used by three different kinds of \ic{Users}: \ic{Organizers}, \ic{Participants} and \ic{Spectators}.\ic{Organizers} will register to Track4Run by filling in all required information in the organizers registration form (see R\subs{39}). \ic{Participants} will enroll in the \ic{Run} through the Data4Help application. \ic{Spectators} may navigate to the \ic{Spectators} dedicated website and select the \ic{Run} they wish to watch from the list. If they are not able to find it, they just need to know the \ic{Run} name or identifier and insert it in the search box at the top of the page (see Section 3.1.1).
				
				\subsubsection{Run Creation and Path Definition}
				\ic{Organizers} have the ability of creating a run. They will be able to give the run a name, set a date and time the run is going to be held on and define a path for it. Moreover, they may limit the number of participants or decide when to close enrollment (see R\subs{41}).
				
				\subsubsection{Display Runners on Map}

				Track4Run will display a map with the real time position of all the \ic{Participants} during a \ic{Run}. \ic{Spectators} may watch a \ic{Run} by inserting its name or identifier. Real time statistics of \ic{Participants} will be shown (e.g. heart rate, rankings). 



\end{document}