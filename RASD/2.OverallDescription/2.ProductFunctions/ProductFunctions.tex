\documentclass[../../rasd.tex]{subfiles}

\begin{document}

\section{Product functions}

	\subsection{Data4Help}
	
		\subsubsection{User Registration}
		Data4Help will allow individuals to register. These will register by entering all the required information (see R\subs{2}). When registering to Data4Help, an individual will first declare to have read the privacy statement and secondly they will have to accept the terms and conditions, which specifically include their consent to the acquisition and processing of their data, including sensitive ones, by TrackMe.\\
		The \ic{User} registration process will be carried out on the Data4Help application (see Section 3.1.1).

		\subsubsection{Third Party Registration}
		A \ic{Third party} may register to Data4Help including all required information (see R\subs{3}).
		Once terms and conditions have been accepted by the \ic{Third party}, it will be successfully registered to Data4Help.

		\subsubsection{User Data Acquisition}
		Data4Help will acquire \ic{User data} through \ic{Smart wearables}. \\
		\ic{Users} must give consent to the acquisition of their data when registering to Data4Help.\\

		\subsubsection{Third Party Data Request}
		Once a \ic{Third party} is registered to Data4Help, it can request access to \ic{Users data} acquired through Data4Help and stored by TrackMe. \ic{Third parties} may request data that refers either to a specific individual - \ic{User data} - or to a group of \ic{Users} identified by common characteristics - \ic{Group data}.\\
		Consent to individual data access is left to the specific \ic{User}, who can either give or deny it to a \ic{Third party} request.\\
		\ic{Group data} will be shared with \ic{Third parties} as long as TrackMe will be able to anonymize it properly (see R\subs{23}).

		\subsubsection{Data Management and Privacy}
		All data acquired through Data4Help will be stored on a database accessible only by TrackMe. At any time, a \ic{User} will be able to revoke the previously given consent to any \ic{Third party} or to TrackMe. Moreover, a \ic{User} may exercise their right to data portability, which means that TrackMe will have to provide them with all the collected data regarding them (see R\subs{32} and R\subs{34}). Finally, \ic{Users} may ask the deletion of all their data stored by TrackMe (see R\subs{33} and R\subs{35}).\\
		By guaranteeing these functions, Data4Help will respect existing general regulations on privacy (e.g. EU GDPR).
	
	\subsection{AutomatedSOS}
		
		\subsubsection{User Subscription}
		All Data4Help \ic{Users} can subscribe to AutomatedSOS through Data4Help application (see Section 3.1.1).
		
		\subsubsection{Health Status Monitoring}
		The \ic{Service} will constantly monitor \ic{User}'s health data to verify if it is \ic{Anomalous}.
		
		\subsubsection{Calling an Ambulance}
		In case the health status of a subscribed \ic{User} is considered not to be good, AutomatedSOS will make a call to local emergency services within 5 seconds asking to send an ambulance to the last registered location of the \ic{User}.

	\subsection{Track4Run}
		
		\subsubsection{User Registration}
		Track4Run will be a \ic{Service} used by three different kinds of users: \ic{Organizers}, \ic{Participants} and \ic{Spectators}. \ic{Organizers} will register to Track4Run by filling in all required information in the organizers registration form (see R\subs{39}). \ic{Participants} will enroll in the \ic{Run} through the Data4Help application. \ic{Spectators} may watch a \ic{Run} looking for it through its name or identifier.
		
		\subsubsection{Run Creation and Path Definition}
		\ic{Organizers} can create a \ic{Run}. They will be able to give the \ic{Run} a name, set a date and time the \ic{Run} is going to be held on and define a path for it. Moreover, they may limit the number of \ic{Participants} or decide when to close enrollment (see R\subs{41} and section 3.1.1).
		
		\subsubsection{Display Runners on Map}
		Track4Run will display a map with the real time position of all the \ic{Participants} during a \ic{Run} (see section 3.1.1). \ic{Spectators} may watch a \ic{Run} by inserting its name or identifier. Real time statistics of \ic{Participants} will be shown (e.g. heart rate, rankings). 

\end{document}