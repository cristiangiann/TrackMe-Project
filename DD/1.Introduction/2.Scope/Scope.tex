\documentclass[../../DD.tex]{subfiles}
\begin{document}
\section{Scope}
	TrackMe proposes to offer its system in a world in continuous technological evolution, where almost everyone owns a smartphone and is always connected to the Internet.
	By using Data4Help, \ic{Users} will integrate their everyday activities with constant collection of their health data and position, through sensors of their \ic{Smart wearable} and smartphone GPS. By adding \ic{Third party Services} they may benefit of data monitoring and analysis with useful insights on their daily activities. \\

	\ic{Users} may enhance data collection by adding AutomatedSOS to their \ic{Services}. This \ic{Service}, by constantly monitoring their data, will allow for \ic{Anomalous data}identification and immediate assistance for the \ic{User in need}. This can be crucial for individuals with health issues and elders living alone.
	People fond of running have the possibility of enrolling in a \ic{Run} listed in Track4Run. Their data, including position and health data, will be shown to \ic{Spectators}, who can watch the \ic{Run}. The process of organizing running competitions will be carried out by \ic{Organizers} through Track4Run.\\

	More details may be found in section 1.2 of  \hyperref[ref:3]{[3]}, including a detailed analysis of shared phenomena.
\end{document}
