\providecommand{\rasd}{..}
\documentclass[../DD.tex]{subfiles}

\begin{document}

\chapter{Introduction}
\thispagestyle{fancy}
		
		\subfile{\rasd/1.Introduction/1.Purpose/Purpose.tex}
		\subfile{\rasd/1.Introduction/2.Scope/Scope.tex}
		\subfile{\rasd/1.Introduction/3.DefinitionsAcronymsAbbreviations/DefinitionsAcronymsAbbreviations.tex}
		
		\section{Revision History}
		\begin{enumerate}
			\item Version 1.0 - 10\sups{th} December 2018
		\end{enumerate}
		
		\section{Reference Documents\label{sect:1.5}}
			\begin{itemize}
				\item Rumbaugh, Jacobson, Booch. 1999. \ic{The Unified Modeling Language Reference Manual}. Addison-Wesley.
				\item \ic{UML-Diagrams}. https://www.uml-diagrams.org/
				\item Rivest, Shamir, Adleman. 1977. \ic{A Method for Obtaining Digital Signatures and Public-Key Cryptosystems}. https://people.csail.mit.edu/rivest/Rsapaper.pdf
			\end{itemize}
			
		\section{Document Structure}
		This document is divided into seven chapters.
		
		\paragraph{Chapter 1}
			This chapter provides a further introduction to the project, in terms of the purpose and the scope, as it has already been introduced in \hyperref[ref:3]{[3]}. Moreover, the chapter contains the definitions, acronyms and abbreviations needed to properly understand the sections following. All the documents used during the development of this project are listed at the end of this chapter.
		\paragraph{Chapter 2}
			This chapter contains a the proposed architecture of the implementation of the system to be described in \hyperref[ref:3]{[3]}. The architecture takes into consideration all the goals and requirements listed in \hyperref[ref:3]{[3]}.\\
			Deployment, sequence, class and other diagrams are provided to allow the reader to have a better understanding of the design of the system. Particular attention is dedicated to non functional requirements satisfaction, including but not limited to reliability and performance. The architecture is described following a top-down approach: the document starts from a general overview, gives a detailed description of component and their interactions and finally provides a detailed class diagram.
		\paragraph{Chapter 3}
			All user interfaces of the system to be may be found in this chapter. These were already provided in section 3.1.1 of \hyperref[ref:3]{[3]} and are further analyzed in this chapter. User interfaces are mapped to components of the system to be that are involved in the user-application interaction. Finally, for each user interface, all the possible actions and results are explained. 
		\paragraph{Chapter 4}
			This chapter is an analysis of which components, described in Chapter 2, allow the satisfaction of each requirement of section 3.2.5 of \hyperref[ref:3]{[3]}. 
		\paragraph{Chapter 5}
			Taking into consideration the architecture and the user interfaces shown in chapters 2 and 3, this chapter provides a list of activities needed to implement the system to be. The activities are analyzed in depth to identify the constraints that force certain activities to be executed before or after others. An implementation, an integration and a testing plan are provided.
		\paragraph{Chapter 6}
			Effort spent by all team members is shown as the list of all activities done during the realization of this document.
		\paragraph{Chapter 7}
			References of documents that this project was developed upon.
		
\end{document}
