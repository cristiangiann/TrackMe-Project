\providecommand{\rasd}{..}
\documentclass[../DD.tex]{subfiles}

\begin{document}

\chapter{Introduction}
\thispagestyle{fancy}
		
		\subfile{\rasd/1.Introduction/1.Purpose/Purpose.tex}
		\subfile{\rasd/1.Introduction/2.Scope/Scope.tex}
		\subfile{\rasd/1.Introduction/3.DefinitionsAcronymsAbbreviations/DefinitionsAcronymsAbbreviations.tex}
		
		\section{Revision History}
		\begin{enumerate}
			\item Version 1.0 - 10\sups{th} December 2018
		\end{enumerate}
		
		\section{Reference Documents}
			\begin{itemize}
				\item \todo{WORK IN PROGRESS}
			\end{itemize}
			
		\section{Document Structure}
		This document is divided into seven chapters.
		
		\paragraph{First chapter}
		In this chapter is summarized the project. This has been already introduced and described in RASD so here it is shown briefly only in terms of purpose and scope. Moreover, the chapter contains the definitions, acronyms and abbreviations needed to properly understand the sections following. All the documents used during the development of this project are listed at the end of the chapter.
		\paragraph{Second chapter}
			In this chapter is described a proposal of architecture able to deliver an implementation of the system to be described in the RASD. The architecture takes into consideration all the goals and requirements listed in the RASD. A particular attention is dedicated to the satisfaction of non functional requirements (e.g. reliability, performances). The architecture is described following a top-down approach: from overview to detailed components descriptions and interaction flows.
		\paragraph{Third chapter}
			In this chapter are listed and showed all the user interfaces of the system to be. The description is given with a particular attention for details and interaction between different interfaces.
		\paragraph{Fourth chapter}
			In this chapter requirements defined in the RASD are mapped to the design elements defined in in Chapter 2.
		\paragraph{Fifth chapter}
			In this chapter is presented the list of activities needed to implement the system to be, according to the architecture showed in Chapter 2 and according the user interfaces showed in Chapter 3. Activities are analyzed to identify eventual constraints that force certain activities to be executed before or after other else. \todo{A rough estimation and scheduling is given.}
		\paragraph{Sixth chapter}
			Effort spent by all team members is shown as the list of all activities done during the realization of this document.
		\paragraph{Seventh chapter}
			References of documents that this project was developed upon.
		
\end{document}
