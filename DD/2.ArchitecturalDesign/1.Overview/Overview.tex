\documentclass[../../DD.tex]{subfiles}
\begin{document}
\section{Overview \label{sect:2.1}}
	%%%%%%%%%%%%%%%%%%%%%%%%%%%%%%%%%%%%%%% DATA4HELP OVERVIEW %%%%%%%%%%%%%%%%%%%%%%%%%%%%

	\image {13cm} {Overviews/OverviewData4Help.jpg} {Data4Help Overview} {OverviewData4Help}

	This diagram is a general overview of the system to be.\\
	The \ic{User} interacts with their smartphone performing several actions, including but not limited to managing their services, sending their consent, and using a service. The smartphone is constantly connected to the Data4Help Server both calling it through its REST APIs. If requested by the third party, Data4Help can enstablish a socket connection between itself and specific users to collect data at the maximum throughput frequency enabled by the users' smartwearables. This feature it's used for example by \ic{Track4Run} during a \ic{Run} or by \ic{AutomatedSOS} when a \ic{User} has parameters that are getting near to the alarm range. The smartphone is also connected to the \ic{User's Smart wearable}, allowing for continuous data collection.\\
	The Data4Help Server is the core of the system to be. It manages all \ic{User} requests, collects its data and sends them to the database - Data4Help DB Server. It also sends through a socket connection newly collected \ic{User data} only to those \ic{Services Users} are subscribed to. It allows the forwarding of \ic{Third Party User data} requests to \ic{Users} and supports the interaction of \ic{Third Parties} with the system through their dedicated website.
	Lastly, the Third Party Service System communicates with the \ic{User's} smartphone via HTTP. Data4Help offers the possibility of embedding the Third Party Service website into the Data4Help app to enable users using custom interfaces for every Service. Third Party has just to provide to Data4Help the address of the website and should develop it in a way that is mobile friendly.
	\image {13cm} {Overviews/OverviewAutomatedSOS.jpg} {AutomatedSOS Overview} {OverviewAutomatedSOS}
	AutomatedSOS is based on a central server which receives \ic{User data} as soon as it is produced and constantly analyzes it. The data is sent from Data4Help to AutomatedSOS via a socket connection always active. When AutomatedSOS recognizes a \ic{User} as a \ic{User in need}, it calls Local Emergency Services and store in the AutomatedSOS database all the relevant information, including which which Local Emergency Services are in charge of assisting a specific \ic{User in need}.
	%%%%%%%%%%%%%%%%%%%%%%%%%%%%%%%%%%%%%%% TRACK4RUN OVERVIEW %%%%%%%%%%%%%%%%%%%%%%%%%%%%
	\image {13cm} {Overviews/OverviewTrack4Run.jpg} {Track4Run Overview} {OverviewTrack4Run}
	The overview of Track4Run gives a general idea of the several components that comprise this \ic{Service}.\\
	The Track4Run Server manages all the actions that can be performed by the authors in this context. First of all, it allows \ic{Organizers} to create a \ic{Run} and manage it through a dedicated website. Moreover, it communicates with Data4Help in order to allow \ic{Users} to enroll in a \ic{Run} as \ic{Participants} and acquire their real time data while running. This data will be displayed in the dedicated \ic{Spectators} website, together with the real time position of \ic{Participants}, retrieving the real world map of the \ic{Run} through Google Maps. All data that needs to be stored, is passed to and retrieved from a database - DB Server. This data is mainly made up of \ic{Run} information.

	

	\end{document}
