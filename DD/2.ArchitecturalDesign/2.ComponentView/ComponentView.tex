\documentclass[../../DD.tex]{subfiles}
\begin{document}
\section{Component View\label{sect:2.2}}

	\image {13cm} {Component_Diagrams/ComponentData4HelpView.jpg} {Data4Help Component Diagram Overview} {ComponentData4HelpOverview}

	A general view of the main components of the system to be is given. The core components are the Data4Help User Application and the server side Application - Data4Help Server. They communicate through three different interfaces offered by Data4Help. These will be further explained in the following two diagrams. Here they are simplified under the Data4Help Interface. The Data4Help User Application receives data from the Smart Wearable through the Data Interface and pushes them to Data4Help. It uses the Service Interface to retrieve the \ic{Service} website and interact with it.\\
	The Data4Help Server provides an interface for the Third Party Service System to send data requests. It also sends newly collected data to the Third Party Service System through the New Data Interface. It finally communicates with the database through the offered Database Interface.

	\image {13cm} {Component_Diagrams/ComponentData4Help.jpg} {Data4Help Component Diagram} {ComponentData4Help}

	These are the internal components and interfaces of the Data4Help server side Application, together with the interfaces offered to external components.\\
	The Authentication component is needed to authenticate the \ic{User} through the User Application. It needs to access the database where all login information is stored through the Database Interface.
	The Data Receiver is needed to receive data from the User Application as soon as it is collected. It then sends it over to the Data Manager through the Data Interface, whose function is to manage all \ic{User data} stored by Data4Help. In fact, it uses the Database Interface to query it.\\
	The Data Manager has a crucial role when it comes to \ic{Third party} requests. A \ic{Third party} may send a request through the Request Data Interface. The request is processed by the Third Party Request Manager. It uses the Add Notifications Interface when a \ic{User} has not yet given consent to a \ic{Third party Service} and the \ic{Third party} wants to acquire its data. It also uses the Data Interface of the Data Manager to retrieve \ic{User data} and \ic{Group data}. The interaction of this component with the others is further explained in section \hyperref[sect:2.4]{2.4}.\\
	Furthermore, the Notifications Manager is needed to provide notifications to the User Application. In fact, the application constantly queries the Data4Help Server for new notifications through the Notifications Interface. The Notifications Manager interacts with the database to retrieve all notifications for a given \ic{User}.\\
	Finally, the Services Manager implements the Services Interface, which allows \ic{Users} to add new \ic{Services} or manage the ones they already have through the User Application. Moreover, it uses the New Data Interface offered by the Third Party Service System when it needs to send new \ic{User data} or \ic{Group data} to \ic{Third parties}.

	\image {13cm} {Component_Diagrams/ComponentUserAppData4Help.jpg} {Data4Help User Application Component Diagram} {ComponentUserAppData4Help}

	The Data4Help User Application is used by \ic{Users} to interact with the Data4Help Server and the \ic{Services} they added. They can do this through several components and interfaces. They must be authenticated in order to use the Application: through the Authentication component, they can sign up and login into their Data4Help account. Moreover, the User Application will then use the Registration and Login Interface offered by the server side Application to authenticate into it.
	The User Application collects \ic{User data} from both the smartphone and the \ic{Smart wearable} through the Data Interface of the Data Collector. This forwards the data to Data4Help through the Data Interface offered by the Data Sender which sends them to Data4Help.\\
	The Services Manager allows \ic{Users} to add and manage their \ic{Services}. This component exploits the Services Interface offered by Data4Help to accomplish its main functions. Moreover, it is in charge of constantly asking Data4Help for new notifications through the Notifications Interface.\\
	Furthermore, the purpose of the Service Controller is to allow the interaction with a \ic{Service} on the User Application through the Service Interface. In fact, the Third Party System will offer this interface for the User Application to connect and interact with. Finally, the Service Controller connects to Google Maps to obtain maps information for \ic{Services} like Track4Run.

	\image {13cm} {Component_Diagrams/ComponentAutomatedSOS.jpg} {AutomatedSOS Component Diagram} {ComponentAutomatedSOS}

	The main component of AutomatedSOS Server is the User Monitor. It offers the AutomatedSOS Commands Interface for the Service Interface Provider to forward commands received from the interaction with the User Application through the Service Interface. In particular, it allows \ic{Users} to reactivate their data monitoring. This component also uses the Database Interface to store which Local Emergency Services are in charge of assisting a specific \ic{User in need}.\\
	When new data is received from Data4Help, through the New Data Interface, it is passed to the Data Analyzer. The Data Analyzer compares the \ic{User data} against certain thresholds, possibly identifying it as \ic{Anomalous data}. When a \ic{User} is identified as a \ic{User in need}, the User Monitor sends a call request through the Caller Interface to the Local Emergency Services Caller, which calls the Local Emergency Services.

	\image {13cm} {Component_Diagrams/ComponentTrack4Run.jpg} {Track4Run Component Diagram} {ComponentTrack4Run}

	At the core of Track4Run lies the Run Manager. It is responsible for everything that concerns a \ic{Run} and it offers several interfaces. First of all, it offers the Track4Run Commands Interface, for the Service Interface Provider to forward commands received from the interaction with the User Application through the Service Interface. In particular, it gives \ic{Users} the possibility to enroll in a \ic{Run} as \ic{Participants} and to lookup available \ic{Runs}. The Run Manager also gives \ic{Organizers} the possibility to create and manage a \ic{Run} through the Run Management Interface from the \ic{Organizers} dedicated website. It uses the Database Interface to store relevant data in the Track4Run  database. Moreover, it offers the Spectators Interface to \ic{Spectators} to watch a \ic{Run} through the \ic{Spectators} dedicated website.\\
	The Authentication allows \ic{Organizers} to sign up and login in the dedicated website through the Registration and Login interface.
	Finally, the Data Receiver offers the New Data Interface, since all \ic{Services} need to do so as to receive new data from Data4Help. It forwards the just received data to the Run Manager through the Data Interface, which is in charge of displaying it to \ic{Spectators} and of storing it in the database.
	
\end{document}