\documentclass[../../DD.tex]{subfiles}
\begin{document}
\section{Selected Architectural Styles}
	\paragraph{Three Tier Architecture}\mbox{}\\
	For all the three systems to be developed, the chosen architecture is a three tier one. In fact, Data4Help, AutomatedSOS and Track4Run, all need three main components: one or more clients, a server and a database, in a remote presentation pattern.\\
	In the remote presentation pattern the client is only responsible of providing a GUI to the user for them to interact with the server. Both Data4Help and \ic{Third party Services}, in fact, need their users to interact with the application. For instance, a \ic{User} may add a new \ic{Service} or they may enroll in a \ic{Run}. This architectural style was adopted for several reasons:
	\begin{itemize}
		\item\textbf{Performance} First of all, the decision of not having any logic on the client side allows to manage the processes only on the server side. This is crucial in order not to have a heavy impact on the user's smartphone performance.
		\item\textbf{Maintainability} This decision also has an effect on maintainability, since the application will not need any updates if a logic change is required in the future: only the server will be updated.
		\item\textbf{Platform independence} By doing this, if \ic{Third party Services} already have a GUI in the form of a website, they do not need to implement a different GUI for the user to use: the \ic{Service} website will be displayed on the client application. The GUI they offer is accessible from any operating system, since it is a website, with no need to have platform specific GUIs.
	\end{itemize}
	The server is the core of the system. For all three systems, it manages all the processes and the logic. It queries the database and it does not need any GUI. All users querying the server have a dedicated application, in the form of a website or of a smartphone application. For instance, the \ic{Organizers} manage \ic{Runs} from the \ic{Organizers} website.\\
	The database is where all information is stored. The decision of not having a local database for the client application is a consequence of the fact that there is no need of having it since all data can be retrieved through the network passing from the server.

	\paragraph{Communication between Components}\mbox{}\\
	This paragraph makes references to elements shown in the diagrams of section \hyperref[sect:2.1]{2.1}.\\
	Data4Help is based on constant data exchange from the client application to the server. In order to allow this continuous communication, a socket connection is established between client and server. This allows for a continuous data stream. Therefore, Data4Help can receive data as soon as it is collected by the client application. This kind of connection is present also between Data4Help and the Third Party Service Systems for them to receive new data as soon as it is produced as long as they are subscribed to it.\\
	Data4Help Server offers REST APIs to both the client application and the Third Party Service Systems. The client application will use the REST APIs to interact with the server. \ic{Third parties} will receive the documentation relative to the exposed APIs for them to communicate with Data4Help. By doing this, \ic{Third parties} may develop their system, having in mind what operations they can do on the server. Moreover, there is no need of a particular communication protocol, as REST APIs are just basic HTTP calls exploiting the main operations: GET, POST, PUT, DELETE.\\
	The communication between the client application and the Third Party System Service is done through HTTP, as already explained in the previous paragraph.

	\paragraph{Track4Run Users}\mbox{}\\
	Track4Run has three kinds of users: \ic{Spectators}, \ic{Organizers} and \ic{Participants}. These three all have different needs and interactions with the system, reason why they are kept separate. Each one has a different way to interact with the system:
	\begin{itemize}
		\item\textbf{Spectators} are the simplest kind of user. They only need a dedicated GUI where they can watch a \ic{Run}.
		\item\textbf{Organizers} require a more complex GUI to interact with the system, allowing for registration and \ic{Run} management.
		\item\textbf{Participants} are Data4Help \ic{Users}, who, after adding Track4Run to their \ic{Services}, may enroll in a \ic{Run}. 
	\end{itemize}\mbox{}\\
	For these reasons, \ic{Spectators} and \ic{Organizers} do not need to be Data4Help \ic{Users}: in fact, their data collection is not needed. By doing this, the system is kept simple by not requiring \ic{Organizers} to register to Data4Help and \ic{Spectators} not to register at all.

	\paragraph{Data Interface}\mbox{}\\
	When it comes to sending data from a component to another component, then it is the receiving component that offers the interface to receive data. This choice is due to the fact that as soon as data is produced, it must be sent over to the receiving component, assuming it is always ready for data receipt. If it were the opposite - where the interface would be offered by the sending component - then the receiving component would have to constantly query the sending component for new data. This solution would have as a consequence several useless queries. On the other hand, all the calls to the data interface are useful.

	\paragraph{Notifications}\mbox{}\\
	The pattern used to implement the notifications system is the opposite of the one discussed in the previous paragraph. In fact, it is the client application that constantly queries the server for new notifications. As soon as they are produced, the server will return them. This choice is driven by the fact that the client application is not always connected to the Internet, therefore, in an opposite scenario, all the calls from the server to the client application, when the latter is offline, would be useless as they will not have any answer. By implementing the chosen solution, all calls will be useful, even if many may still return an empty list of notifications.

	\paragraph{Third Party Services Interfaces}\mbox{}\\
	With the proposed design, there are only two different interfaces that \ic{Third party Services} need to implement and offer. These are the Service Interface and the New Data Interface (see sections \hyperref[sect:2.2]{2.2} and \hyperref[sect:2.5]{2.5} for more details).
	The two interfaces are here analyzed in details:
	\begin{itemize}
		\item \textbf{Service Interface} must respect the documentation provided by TrackMe in order to allow the User Application to retrieve the GUI offered by the \ic{Service}. In particular, the GUI must be a website for the \ic{User} to interact with. The User Application will call methods of the interface, therefore this must be \ic{Service}-independent. A solution to this lies in the executeCommand(String command) method, where the User Application will pass a string containing the command to be executed to the \ic{Service} offering the Service Interface.
		\item \textbf{New Data Interface} must respect the documentation provided by TrackMe. This interface is crucial for \ic{Third party Services} to receive new data. In fact, when new data is collected, it is sent to the subscribed \ic{Third party Services} right away. In order to have a \ic{Service}-independent way of sending new data, a common interface must be shared among \ic{Services}.
	\end{itemize}

	\paragraph{Database Interface Design}\mbox{}\\
	The databases of the system to be must not provide a way to execute any query on it. In fact, components have a precise purpose and specific actions they will perform. For each of them, there are methods that they will call to accomplish their functions. This defensive style is crucial to maintain separation between the two tiers - server and database. In fact, the server will not have complete control over the database not being able to execute any query.

\end{document}
