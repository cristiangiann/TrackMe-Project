\documentclass[../../DD.tex]{subfiles}
\begin{document}
\section{Selected Architectural Styles}
	- 3 tier architecture (client server db)
		- Per data4help e services Remote Presentation	 
		- spiegare il discorso di website all'interno dell'applicazione e PERCHE:
			- ogni servizio non deve implementare un app specifica, ma se ha già un website, può usare quello, platform independent
	- Connection and communication
		- Socket for constant data push
		- rest api per chiamate e comandi
		- http per il discorso di sopra
	- in track4run abbiamo deciso di tenere separati i 3 attori perche non ha senso che uno spectator o un organizer siano utenti di d4h, non condividendo i dati
		- spiegare i 3 attori di cosa hanno bisogno (organizer --> website per organizzare / registrazione, spectator solo un'interfaccia)
	- Quando si tratta di inviare dati (data interface) allora è il ricevitore che espone l'interfaccia
		- Appena il dato è prodotto, arriva al ricevitore, che deve essere sempre up
	- Quando si tratta di inviare notifiche, è chi le deve ricevere che chiede se ce ne sono
		- L'app quando si connette chiede al server
	- Ci sono delle interfacce che devono essere implementate dai servizi 3 parti: Service Interface, New Data Interface

\end{document}
